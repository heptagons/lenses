\documentclass[11pt]{article}
\title{Lenses}
\author{https://github.com/heptagons/lenses}
\date{2023/12/29}

\usepackage{graphicx}

\usepackage[margin=0.75in]{geometry}
\usepackage{float} % {figure}{H}
\usepackage{amsmath} % \dfrac


%\usepackage{amssymb}
%\usepackage{subcaption}

\def\mathbi#1{\textbf{\em #1}}

\begin{document}

\maketitle
\begin{abstract}
Lenses are equilateral hexagons resembling concave and convex optical lenses. The hexagons consecutive six internal angles are $(\theta_1,\theta_2,\theta_3,\theta_1,\theta_2,\theta_3)$ where $\theta_1=X\theta_0$, $\theta_2=Y\theta_0$, and $\theta_3=Z\theta_0$ where $\theta_0 = 2\pi/S$ is the base angle of symmetry $S$.
\end{abstract}

\section{Lenses}

\section{Symmetry $5$}

Symmetry $5$ uses as base the angle $\beta = \dfrac{2\pi}5$. Includes two rhombi $\mathbi{b}$ and $\mathbi{c}$ and two lenses $\mathbi{B}$ and $\mathbi{C}$.


\subsection{Rhombi $\mathbi{b}$ and $\mathbi{c}$}

\begin{figure}[H]
\centering
\includegraphics[scale=1.1]{bc/rhombi}
\caption{Rhombi of the types $\mathbi{b}$ and $\mathbi{c}$.}
\label{fig:bc-rhombi}
\end{figure}

Figure \ref{fig:bc-rhombi} show rhombi $\mathbi{b}$ and $\mathbi{c}$. $\mathbi{b}$ is the rhombus with smallest internal angles equal to $\dfrac{\beta}2 = \dfrac{\pi}5$. $\mathbi{c}$ is the rhombus with smallest internal angles equal to $\beta = \dfrac{2\pi}5$.
Figure $(i)$ show a dissected star whose area equals to $10\mathbi{b}$.
Figure $(ii)$ show a dissected star whose area equals to $5\mathbi{c}$.
Figure $(iii)$ show a dissected regular decagon whose area equals to $5\mathbi{b} + 5\mathbi{c}$.


\subsection{Lenses $\mathbi{B}$ and $\mathbi{C}$}

\begin{figure}[H]
\centering
\includegraphics[scale=1.1]{bc/hexagons}
\caption{Lenses of types $\mathbi{B}$ and $\mathbi{C}$.}
\label{fig:bc-hexagons}
\end{figure}

Figure \ref{fig:bc-hexagons} show lenses $\mathbi{B}$ and $\mathbi{C}$.
Figure $(i)$ show the lense $\mathbi{B}$ with perimeter $\overline{B_1...B_6}$ which is formed adding two rhombi $\mathbi{b}$ and adding one rhombus $\mathbi{c}$ so its area equals to $2\mathbi{b} + \mathbi{c}$.
Figure $(ii)$ show the lense $\mathbi{C}$ with perimeter $\overline{C_1...C_6}$ which is formed adding two rhombi $\mathbi{c}$ and substracting one rhombus $\mathbi{b}$ so its area equals to $2\mathbi{c} - \mathbi{b}$.
Figure $(iii)$ show a dissected star whose area equals to $2\mathbi{C} + \mathbi{B} = 5\mathbi{c}$.
Figure $(iv)$ show a dissected regular decagon whose area equals to $3\mathbi{B} + \mathbi{C} = 5\mathbi{b} + 5\mathbi{c}$.



\end{document}
